%%%%%%%%%%%%%%%%%%%%%%%%%%%%%%%%%%%%%%%%%%%%%%
% An example of a lab report write-up.
%%%%%%%%%%%%%%%%%%%%%%%%%%%%%%%%%%%%%%%%%%%%%%
% This is a combination of several labs that I have done in the past for
% Computer Engineering, so it is not to be taken literally, but instead used as
% a great starting template for your own lab write up.  When creating this
% template, I tried to keep in mind all of the functions and functionality of
% LaTeX that I spent a lot of time researching and using in my lab reports and
% include them here so that it is fairly easy for students first learning LaTeX
% to jump on in and get immediate results.  However, I do assume that the
% person using this guide has already created at least a "Hello World" PDF
% document using LaTeX (which means it's installed and ready to go).
%
% My preference for developing in LaTeX is to use the LaTeX Plugin for gedit in
% Linux.  There are others for Mac and Windows as well (particularly MikTeX).
% Another excellent plugin is the Calc2LaTeX plugin for the OpenOffice suite.
% It makes it very easy to create a large table very quickly.
%
% Professors have different tastes for how they want the lab write-ups done, so
% check with the section layout for your class and create a template file for
% each class (my recommendation).
%
% Also, there is a list of common commands at the bottom of this document.  Use
% these as a quick reference.  If you'd like more, you can view the "LaTeX Cheat
% Sheet.pdf" included with this template material.
%
% (c) 2009 Derek R. Hildreth <derek@derekhildreth.com> http://www.derekhildreth.com
% This work is licensed under the Creative Commons Attribution-NonCommercial-ShareAlike License. To view a copy of this license, visit http://creativecommons.org/licenses/by-nc-sa/1.0/ or send a letter to Creative Commons, 559 Nathan Abbott Way, Stanford, California 94305, USA.
%%%%%%%%%%%%%%%%%%%%%%%%%%%%%%%%%%%%%%%%%%%%%%
\documentclass[aps,letterpaper,10pt]{revtex4}

\input kvmacros % For Karnaugh Maps (K-Maps)


\usepackage{multirow}
\usepackage{graphicx} % For images
\usepackage{float}    % For tables and other floats
\usepackage{verbatim} % For comments and other
\usepackage{amsmath}  % For math
\usepackage{amssymb}  % For more math
\usepackage{fullpage} % Set margins and place page numbers at bottom center
\usepackage{listings} % For source code
\usepackage{subfig}   % For subfigures
\usepackage[usenames,dvipsnames]{color} % For colors and names
\usepackage[pdftex]{hyperref}           % For hyperlinks and indexing the PDF
\usepackage{xcolor}

\hypersetup{ % play with the different link colors here
    colorlinks,
    citecolor=blue,
    filecolor=blue,
    linkcolor=blue,
    urlcolor=blue % set to black to prevent printing blue links
}


\lstdefinestyle{lfonts}{
basicstyle = \footnotesize\ttfamily,
stringstyle = \color{purple},
keywordstyle = \color{blue!60!black}\bfseries,
commentstyle = \rmfamily
}
\lstdefinestyle{lnumbers}{
numbers = left,
numberstyle = \tiny,
numbersep = 1em,
firstnumber = 1,
stepnumber = 1,
}
\lstdefinestyle{llayout}{
breaklines = true,
tabsize = 2,
columns = flexible,
}
\lstdefinestyle{lgeometry}{
xleftmargin = 20pt,
xrightmargin = 0pt,
frame = tb,
framesep = \fboxsep,
framexleftmargin = 20pt,
}
\lstdefinestyle{lgeneral}{
style = lfonts,
style = lnumbers,
style = llayout,
style = lgeometry,
}
\lstdefinestyle{python}{
language = {Python},
style = lgeneral,
}

\lstset{
    basicstyle          =   \sffamily,          % 基本代码风格
    keywordstyle        =   \bfseries,          % 关键字风格
    commentstyle        =   \rmfamily\itshape,  % 注释的风格,斜体
    stringstyle         =   \ttfamily,  % 字符串风格
    flexiblecolumns,                % 别问为什么,加上这个
    numbers             =   left,   % 行号的位置在左边
    showspaces          =   false,  % 是否显示空格,显示了有点乱,所以不现实了
    numberstyle         =   \zihao{-5}\ttfamily,    % 行号的样式,小五号,tt等宽字体
    showstringspaces    =   false,
    captionpos          =   t,      % 这段代码的名字所呈现的位置,t指的是top上面
    frame               =   lrtb,   % 显示边框
}



% Make units a little nicer looking and faster to type
\newcommand{\Hz}{\textsl{Hz}}
\newcommand{\KHz}{\textsl{KHz}}
\newcommand{\MHz}{\textsl{MHz}}
\newcommand{\GHz}{\textsl{GHz}}
\newcommand{\ns}{\textsl{ns}}
\newcommand{\ms}{\textsl{ms}}
\newcommand{\s}{\textsl{s}}
\newcommand{\RNum}[1]{\uppercase\expandafter{\romannumeral #1\relax}}


% TITLE PAGE CONTENT %%%%%%%%%%%%%%%%%%%%%%%%
% Remember to fill this section out for each
% lab write-up.
%%%%%%%%%%%%%%%%%%%%%%%%%%%%%%%%%%%%%%%%%%%%%
\newcommand{\labno}{05}
\newcommand{\labtitle}{AU 332 Artificial Intelligence: Principles and Techniques}
\newcommand{\authorname}{Zhengbao He(517030910157)}
\newcommand{\hw}{1}
% END TITLE PAGE CONTENT %%%%%%%%%%%%%%%%%%%%


\begin{document}  % START THE DOCUMENT!


% TITLE PAGE %%%%%%%%%%%%%%%%%%%%%%%%%%%%%%%%%%%%%%
% If you'd like to change the content of this,
% do it in the "TITLE PAGE CONTENT" directly above
% this message
%%%%%%%%%%%%%%%%%%%%%%%%%%%%%%%%%%%%%%%%%%%%%%%%%%%
\begin{titlepage}
\begin{center}
{\Large \textsc{\labtitle} \\ \vspace{4pt}}
\rule[13pt]{\textwidth}{1pt} \\ \vspace{150pt}
{\large By: \authorname \\ \vspace{10pt}
HW\#: \hw \\ \vspace{10pt}
\today}
\end{center}
\end{titlepage}
% END TITLE PAGE %%%%%%%%%%%%%%%%%%%%%%%%%%%%%%%%%%





%%%%%%%%%%%%%%%%%%%%%%%%%%%%%%
%%%%%%%%%%%%%%%%%%%%%%%%%%%%%%
\section{Introduction}
%No Text Here
%%%%%%%%%%%%%%%%%%%%%%%%%%%%%%%
\subsection{Purpose}
\begin{comment}
This is a lab template which has a ton of different things which are useful in writing lab write-ups in the Computer Eningeering field.
This is demonstrating the comment block. 
Don't be overwhelmed, it may seem like a lot to take in at a time, but it's worth spending the time learning it.
\end{comment}
In this homework, we are going to apply minimax search with alpha-beta pruning on Chinese Checkers.
Through this assignment, we will learn how to design and implement a Minimax algorithm 
and try to optimize it using alpha-beta pruning, a better evaluation function and other methods with limited computing resources.
Finally, we conducted a simple assessment of this assignment to deepen our understanding of adversarial search.

\vspace{3mm} % I use this to seperate the paragraphs a bit.

%%%%%%%%%%%%%%%%%%%%%%%%%%%%%%
\subsection{Equipment}
I finished the homework with:
	\begin{itemize}
		\item A laptop with Windows 10 or Linux
		\item Anaconda 3
		\item VS Code
		\item TexLive
	\end{itemize}

%%%%%%%%%%%%%%%%%%%%%%%%%%%%%%


%%%%%%%%%%%%%%%%%%%%%%%%%%%%%%
%%%%%%%%%%%%%%%%%%%%%%%%%%%%%%


%%%%%%%%%%%%%%%%%%%%%%%%%%%%%%
%%%%%%%%%%%%%%%%%%%%%%%%%%%%%%
\newpage
\section{Code and algorithm}
This section will introduce the code and algorithm we design and some attempts we made(even not effective).
\vspace{3mm}


% IF YOU'D RATHER TYPE THE CODE, OR HAVE A SMALLER BLOCK OF CODE, USE THIS:
%\begin{lstlisting}
%if(something)
%	do this
%else
%	do this
%\end{lstlisting}

%% THIS IS FROM A DIFFERENT CLASS, BUT DEMONSTRATES MATH MODE WELL
%%%%%%%%%%%%%%%%%%%%%%%%%%%%%%
\subsection{The basic Minimax algorithm with alpha-beta pruning}

\begin{lstlisting}[style = python]
def MiniMax_pruned_version(self, state, depth, al, be, Depth=2):
	if depth != Depth:
		alpha = al
		beta = be
		#print('\t'*depth + "第{}层节点生成,alpha = {},beta = {}".format(depth, alpha, beta))
		if depth % 2 == 0:  # max layer
			evaluation = -10000
		else:
			evaluation = 10000
		selected_action = None
		legal_actions = self.game.actions(state)
		player = state[0]
		for action in self.stimulation_max(state, legal_actions):
			if action[1][0]*(-1)**player < action[0][0]*(-1)**player:
				continue
			board = state[1]
			board.board_status[action[0]] = 0
			board.board_status[action[1]] = player
			value, next_action = self.MiniMax_pruned_version((3 - player, board), depth + 1, al=alpha, be=beta)
			board.board_status[action[0]] = player
			board.board_status[action[1]] = 0
			if depth % 2 == 0:  # max layer
				if value > evaluation:
					evaluation = value
					selected_action = action
					alpha = value
			else:  # min layer
				if value < evaluation:
					evaluation = value
					selected_action = action
					beta = value
			if alpha >= beta:
				#print('\t'*depth + "Pruned in layer",depth)
				return evaluation, selected_action
		return evaluation, selected_action
	else:
		evaluation_value = self.evaluation(state, p=2, i=10, d=5)
		return evaluation_value, None
\end{lstlisting}[style = python]

The code above implements a simple MiniMax algorithm with alpha-beta pruning and the default depth is 2,
which means that we only consider two movement of our chess.
This is because when the two pieces are entangled, considering too many moves makes the algorithm very complicated. 
What we are thinking about is how to reach the end point faster than the enemy, and the enemy is also.
Because both parties will give priority to their own movements, too deep search will introduce more uncertainty and will consume more resources.
So, when the two players’ pieces are not separated, we just use the MiniMax algorithm whose depth is 2.


\subsection{SelfMax algorithm}
\begin{lstlisting}[style=python]
def selfMax(self, state, depth, Depth=2):
	if depth != Depth:
		evaluation = -10000
		selected_action = None
		legal_actions = self.game.actions(state)
		player = state[0]
		for action in self.stimulation_max(state, legal_actions):
			if action[1][0] * (-1) ** player < action[0][0] * (-1) ** player:
				continue
			board = state[1]
			board.board_status[action[0]] = 0
			board.board_status[action[1]] = player
			ver_positions = [position[0] for position in board.getPlayerPiecePositions(player)]
			if sum(ver_positions) == (2 - player) * 30 + (player - 1) * 170:
				board.board_status[action[0]] = player
				board.board_status[action[1]] = 0
				return 10000, action
			value, next_action = self.selfMax((player, board), depth + 1)
			board.board_status[action[0]] = player
			board.board_status[action[1]] = 0
			if value > evaluation:
				evaluation = value
				selected_action = action
		return evaluation, selected_action
	else:
		evaluation = self.evaluation(state, p=1, i=5, d=0)
		return evaluation, None
\end{lstlisting}[style=python]

When two players’ pieces are separated, we apply another algorithm which called SelfMax algorithm.
It just considers the movement of us, without considering that of the opponent, which will not affect our chess pieces.
So in this case, considering opponent's movement only consume computing resources.
Bacause we don't need to consider the opponent's movement, the two factors limiting depth in MiniMax above disappear,
we can deepen the search depth and get better performence.

\subsection{Evaluation Function}
\begin{lstlisting}[style = python]
def evaluation(self, state, p, i, d):
	player = state[0]
	board = state[1]

	if player == 1:
		ver_positions = [position[0] for position in board.getPlayerPiecePositions(player)]
		op_p = [position[0] for position in board.getPlayerPiecePositions(3 - player)]
		hor_positions = [abs(position[1] - board.getColNum(position[0]) / 2) / board.getColNum(position[0]) for
							position in board.getPlayerPiecePositions(player) if position[0] % 2 == 0]

		ver_displacement = sum(ver_positions)
		lagger = 2 * board.size - 1 - max(ver_positions)
		hor_displacement = sum(hor_positions)
	else:
		ver_positions = [2 * board.size - 1 - position[0] for position in board.getPlayerPiecePositions(player)]
		op_p = [2 * board.size - 1 - position[0] for position in board.getPlayerPiecePositions(3 - player)]
		hor_positions = [abs(position[1] - board.getColNum(position[0]) / 2) / board.getColNum(position[0]) for
							position in board.getPlayerPiecePositions(player) if position[0] % 2 == 0]
		ver_displacement = sum(ver_positions)
		lagger = 2 * board.size - 1 - max(ver_positions)
		hor_displacement = sum(hor_positions)

	return - p * ver_displacement + i * lagger - d * hor_displacement - p * sum(op_p)
\end{lstlisting}[style = python]

When the search depth is limited because of limited computing resources, it is very important to evaluate the current game when the outcome is not divided.
In the evaluation function above, we use four factors to evaluate the move decision:
\begin{itemize}
	Total vertical distance of our pieces
	The distance of the piece from the center line of the board
	The lagger piece--the piece at the end of the board
	Total vertical distance of opponent's pieces
\end{itemize}

It's obvious that we should use the total vertical distance to evaluate the game because the smaller vertical distance means more possible to win.
And it's also better to stay close to the center line of the board. 
When a piece is on the edge of the board, it has less chances to interact with other pieces,
which means less possible to make a long hop.
And we also consider the "lagger" piece. If one piece is left behind by other pieces, it's more likely to move step by step.
The last factor we consider is the total vertical distance of opponent's pieces. 
The reason is mentioned above: our opponent gives priority to his own movement. 
If we don't consider our opponent, the "MIN" in MiniMax is meaningless.



%%%%%%%%%%%%%%%%%%%%%%%%%%%%%%
\subsection{A* algorithm}

\vspace{3mm}


There are two points to note about this specific algorithm used in the solution:

First, because of the existence of the 'wall', the Manhattan distance does not work well as a heuristic function, although it satisfies consistency. 
In this question, I use another heuristic function, specifically: 

\vspace{1mm}
\emph{
when the current node is in the upper part of the graph(its row value is less than or equal to 2), 
its \emph{h} value is The sum of the Manhattan distances of it to node (0,3) and node (0,3) to the goal node. 
Correspondingly, when it is in the lower half of the graph, its \emph{h} value is The sum of the Manhattan distances of it to node (4,3) and node (4,3) to the goal node. }
\vspace{1mm}

It is easy to prove that this heuristic function also satisfies the consistency.
\vspace{2mm}

Second, when two nodes have the same \emph{g} value(\emph{g = f + h,f} represents the cost of a node),
it's obvious that you should first expand the node with a larger cost, that is, a node with a smaller \emph{h} value. 
Because the smaller value of \emph{h}(the larger cost) means this node may be closer to the goal node.

\vspace{3mm}
Based on the above two points, I solved this problem using the A* algorithm.

\newpage
\subsection{Programming Assignment}

Please check the code in \emph{BFSvsDFS.py, UniformCostSearch.py \emph{and} AStarSearch.py.}

It is worth noting that in the code editing area of the search algorithm in each file, 
I have called the function in \emph{E.1} to check if the goal and start state are valid nodes in the graph.
So the output may be a little different.You can learn more in \emph{E.1}.

\vspace{3mm}

\subsection{Extra credit}
\noindent
Question:

For the programming assignment part, extra credits will be given if you completed the following cases.

\qquad 1. Check if the goal and start state are valid nodes in the graph, return error handling message.

\qquad 2. Check whether the graph satisfies the consistency of heuristics.
\vspace{3mm}

For E.1, the function is below:
\begin{lstlisting}[style = python]
def check_valid(graph, start, goal):
	print('Begin to check the validity of the start node %s and the goal node %s in the graph:' % (start, goal))
	flag_of_start = False # True means that start is a valid node, while False means not.
	flag_of_goal = False # True means that goal is a valid node, while False means not.
	for i in graph.edges.keys():
		if i==start or start in graph.edges[i]:
			flag_of_start = True
		if i==goal or goal in graph.edges[i]:
			flag_of_goal = True
		if flag_of_start and flag_of_goal:
			break
	print("\tThe start node is%s a valid node." % '' if flag_of_start else ' not')
	print("\tThe goal node is%s a valid node." % '' if flag_of_goal else ' not')
	return flag_of_start, flag_of_goal
\end{lstlisting}

The first return value of the function represents whether the start state is a valid node, 
and the second return value represents whether the goal state is a valid node.

I also call the function in the code editing area of search algorithms \emph{BFSvsDFS.py, UniformCostSearch.py \emph{and} AStarSearch.py.}\vspace{3mm}

For E.2, I also wrote a function as following:
\begin{lstlisting}[style = python]
	def check_graphs_consistency_of_heuristics(graph,goal):
    	not_satisfied=[]
    	for i in graph.edges.keys():
        	for j in graph.edges[i]:
            	if (heuristic(graph,i,goal)-heuristic(graph,j,goal))>graph.get_cost(i,j):
                not_satisfied.append((i,j))
    	if not_satisfied == []:
        	print('The graph satisfies the consistency of heuristics.')
        	return True
    	else:
        	print('The graph doesn\'t satisfi	y the consistency of heuristics because:')
        	for i,j in not_satisfied:
            	print('The h value of %s is %2.4f and the h value of %s is %2.4f but the cost from %s to %s is %d'%
            	(i,heuristic(graph,i,goal),j,heuristic(graph,j,goal),i,j,graph.get_cost(i,j)) )
        return False
\end{lstlisting}

\newpage
The return value and the printed message both indicate whether the graph satisfies the consistency of heuristics.
I called this function in \emph{AStarSearch.py}, and the printed message is below:
\begin{lstlisting}[style = python]
	'''
	The small graph doesn't satisfiy the consistency of heuristics because:
		The h value of B is 7.2111 and the h value of D is 2.8284 but the cost from B to D is 4
		The h value of C is 8.0000 and the h value of A is 5.6569 but the cost from C to A is 2
	The large graph doesn't satisfiy the consistency of heuristics because:
		The h value of S is 10.0000 and the h value of B is 7.8102 but the cost from S to B is 2
		The h value of B is 7.8102 and the h value of H is 5.0000 but the cost from B to H is 1
		The h value of C is 8.0000 and the h value of L is 5.0990 but the cost from C to L is 2
		The h value of G is 3.1623 and the h value of E is 0.0000 but the cost from G to E is 2
	'''
\end{lstlisting}

The results show that the heuristic function I designed does not satisfy the consistency in both figures. 
This is not only related to functions, but also to the nature of the graph. 
It is possible that the graphs in \emph{AStarSearch.py} are not in the European space, so they do not satisfy the consistency. 
If we multiply the Euclidean distance in the heuristic function by a sufficiently small constant, we can make these graphs consistent. 
But when this constant is too small, the heuristic function loses its meaning, and the A* algorithm becomes the UCS algorithm.
Therefore, it is very important to design different heuristic functions for different graphs.

% AGAIN, ANOTHER EXAMPLE FROM A DIFFERENT CLASS WHICH DEMONSTRasdATES KMAPS AND TABLES NICELY.
% I added this after viewing the completed pdf and decided to make this cosmetic change


% This table was generated using the Calc2LaTeX macro which I mentioned earlier.
% You'll need OpenOffice installed and you'll have to download the macro online.
% If you're interested, I have a guide on how to set this up and use it on my
% blog.  http://www.derekhildreth.com/blog  Search for "LaTeX".  You'll find it.


%%%%%%%%%%%%%%%%%%%%%%%%%%%%%%
%%%%%%%%%%%%%%%%%%%%%%%%%%%%%%
\newpage
\section{Discussion \& Conclusion}
After studying these search algorithms, it's obvious that the A* algotithm is the most efficient algotithm.
But when we having no additional information about states beyond that provided in the problem definition, we just can use the uninformed search strategies.

And based on the A* algorithm, we can still make a lot of improvements. 
In addition to the small trick mentioned in \emph{\RNum{2}.C}, we can use dynamic heuristics, or two-way search, etc.
I hope that in the future study, I can continue to deepen the study of these algorithms.


\end{document} % DONE WITH DOCUMENT!


